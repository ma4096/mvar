% example for LaTeX

\documentclass[11pt,a4paper,sans]{article}
\usepackage{amsmath}
\usepackage{longtable} % the longtable is required by the list of abbreviations
% Latex implementation of the variable transfer
% Usage:
% \loadvariables{<namespace>}{<file path>}
% 	<file path> can be (tested on linux):
%		../folder/filename.txt if in another folder (changes directory once and then goes to the folder "folder"
%		filename.txt if in the same folder
% 	<namespace> gives the prefix of all variables from this file
%
%\<namespace><var name>
%	gives the value of the variable with <var name> from the file

%example:
%	given textfile named test.txt in same folder as this file and the main.tex:
% a,45
% b,6.453
%
%	in main.tex:
%\documentclass[11pt,a4paper,sans]{article}
%% Latex implementation of the variable transfer
% Usage:
% \loadvariables{<namespace>}{<file path>}
% 	<file path> can be (tested on linux):
%		../folder/filename.txt if in another folder (changes directory once and then goes to the folder "folder"
%		filename.txt if in the same folder
% 	<namespace> gives the prefix of all variables from this file
%
%\<namespace><var name>
%	gives the value of the variable with <var name> from the file

%example:
%	given textfile named test.txt in same folder as this file and the main.tex:
% a,45
% b,6.453
%
%	in main.tex:
%\documentclass[11pt,a4paper,sans]{article}
%% Latex implementation of the variable transfer
% Usage:
% \loadvariables{<namespace>}{<file path>}
% 	<file path> can be (tested on linux):
%		../folder/filename.txt if in another folder (changes directory once and then goes to the folder "folder"
%		filename.txt if in the same folder
% 	<namespace> gives the prefix of all variables from this file
%
%\<namespace><var name>
%	gives the value of the variable with <var name> from the file

%example:
%	given textfile named test.txt in same folder as this file and the main.tex:
% a,45
% b,6.453
%
%	in main.tex:
%\documentclass[11pt,a4paper,sans]{article}
%\include{MLtransfer.tex}
%\loadvariables{t}{test.txt}
%\begin{document}
%	\tb
%\end{document}
%
%	gives pdf with 6.453


\usepackage{expl3}
\usepackage{xparse}
\usepackage{forloop}
\usepackage{siunitx}


\ExplSyntaxOn
% because we want different ior variables for each instance of loadvariables{...} and ior_str_map_inline doesnt natively accept "variable" names 
\cs_generate_variant:Nn \ior_str_map_inline:Nn { c } 
\cs_generate_variant:Nn \ior_map_inline:Nn { c }
\cs_generate_variant:Nn \tl_if_eq:nnTF { V }

\newcommand{\loadvariables}[2]{
	% this is a dummy command to find the reference to transfer files during precompilation. The real work is being done by \_loadvariables below
}

\newcommand{\backgroundloadvariables}[2]{
	\krishna_csv_load:nn {#1} {#2} % 1: Namespace, 2: Filename/Path
}
% Copied from https://tex.stackexchange.com/questions/474397/populate-information-from-a-csv-file-into-a-latex-document-specifically-into-th/474404#474404
% I dont know enough latex3 to do this myself or do it cleaner :)
%\ior_new:N \l__krishna_ior_tmpa
\cs_new:Npn \krishna_csv_load:nn #1 #2 %#2: Dateiname, #1: Name der CSV intern (Namespace)
{
	\cs_if_free:cTF { l__krishna_ #1 _seq }
	% guckt nach, ob \l__krishna_ #1_seq  noch nicht existiert
	%{ \__krishna_csv_load:nn { l__krishna_ #1 } {#2} } % springt immer hier hin, da immer war
	{ \__krishna_csv_load:nn {#1} {#2} } % springt immer hier hin, da immer war
	{ \msg_error:nnn { krishna / csv } { name-used } {#1} }
}
\cs_new:Npn \__krishna_csv_load:nn #1 #2 %#2: Dateiname, #1: Namespace
{
	\ior_new:c {l__krishna_ior_#1_tmpa} % reserves name of IO-Stream
	\ior_open:cn {l__krishna_ior_#1_tmpa} {#2} % load the File into the IO-Stream
	\seq_new:c {l__krishna_ #1 _seq } % l__krishna_ #1_seq, das was vorher als frei registriert wurde
	\prop_new:c {l__krishna_ #1 _prop} % new
	\seq_new:c {l__krishna_  #1 _seq_unit}
	%\prop_new:c {l__krishna_ #1 _prop_unit}
	%\exp_args:Nc \__krishna_csv_read:N { #1 _seq } % fully expanded
	%\ior_str_map_inline:cn {l__krishna_ior_#1_tmpa}  { \seq_put_right:cn {l__krishna_ #1_seq} {##1} }  % new
	\ior_map_inline:cn {l__krishna_ior_#1_tmpa}  { \seq_put_right:cn {l__krishna_ #1_seq} {##1} }  % new

	% TODO Cant currently access Description or name
	% TODO Change the delimiter to \t here, how?
	\seq_map_inline:cn {l__krishna_ #1 _seq}  { \prop_put:cxx {l__krishna_ #1_prop} {\clist_item:nn {##1}{1}} {\clist_item:nn{##1}{2}} }  % new
	\seq_map_inline:cn {l__krishna_ #1 _seq}  { \prop_put:cxx {l__krishna_ #1_prop_unit} {\clist_item:nn {##1}{1}} {\clist_item:nn{##1}{3}} }


	\ior_close:c {l__krishna_ior_#1_tmpa}
}
\cs_new:Npn \__krishna_csv_read:N #1
{
	%\ior_str_map_inline:Nn \l__krishna_ior_tmpa  { \seq_put_right:Nn #1 {##1} }
}
\cs_new:Npn \krishna_csv_get:nnn #1 #2 #3 % not used
{
	\exp_args:Nf
	\clist_item:nn % comma list WIE AUF \t UMSTELLEN?????
	{ \seq_item:cn { l__krishna_ #1 _seq } {#2} }
	{#3}
}
\msg_new:nnn { krishna / csv } { name-used }
{ The~CSV~name~`#1'~is~already~taken }

\newcommand{\mvar}[2] % #1=namespace, #2=variable
{
	\prop_item:cn {l__krishna_ #1 _prop} {#2}
}
\newcommand{\mvarunit}[2]
{
	\prop_item:cn {l__krishna_ #1 _prop_unit} {#2}
}
% Logic function: if value is 1 do true part, otherwise true
\newcommand{\mvaristrue}[4]  %#1=namespace, #2=variablename, #3=text if true, #4=text if false
{
	\int_set:Nn \l_tmpa_int {\int_eval:n {\prop_item:cn {l__krishna_ #1_prop}{#2}}}
	\int_compare:nNnTF {\l_tmpa_int} = {1} {#3} {#4}
}
\newcommand{\mvarsi}[2] %[3]
{
	%\mvar{#1}{#2}\,\si{#3}
	%\SI{\mvar{#1}{#2}}{#3}
	\SI{\mvar{#1}{#2}}{\mvarunit{#1}{#2}}
}
\ExplSyntaxOff


%\loadvariables{t}{test.txt}
%\begin{document}
%	\tb
%\end{document}
%
%	gives pdf with 6.453


\usepackage{expl3}
\usepackage{xparse}
\usepackage{forloop}
\usepackage{siunitx}


\ExplSyntaxOn
% because we want different ior variables for each instance of loadvariables{...} and ior_str_map_inline doesnt natively accept "variable" names 
\cs_generate_variant:Nn \ior_str_map_inline:Nn { c } 
\cs_generate_variant:Nn \ior_map_inline:Nn { c }
\cs_generate_variant:Nn \tl_if_eq:nnTF { V }

\newcommand{\loadvariables}[2]{
	% this is a dummy command to find the reference to transfer files during precompilation. The real work is being done by \_loadvariables below
}

\newcommand{\backgroundloadvariables}[2]{
	\krishna_csv_load:nn {#1} {#2} % 1: Namespace, 2: Filename/Path
}
% Copied from https://tex.stackexchange.com/questions/474397/populate-information-from-a-csv-file-into-a-latex-document-specifically-into-th/474404#474404
% I dont know enough latex3 to do this myself or do it cleaner :)
%\ior_new:N \l__krishna_ior_tmpa
\cs_new:Npn \krishna_csv_load:nn #1 #2 %#2: Dateiname, #1: Name der CSV intern (Namespace)
{
	\cs_if_free:cTF { l__krishna_ #1 _seq }
	% guckt nach, ob \l__krishna_ #1_seq  noch nicht existiert
	%{ \__krishna_csv_load:nn { l__krishna_ #1 } {#2} } % springt immer hier hin, da immer war
	{ \__krishna_csv_load:nn {#1} {#2} } % springt immer hier hin, da immer war
	{ \msg_error:nnn { krishna / csv } { name-used } {#1} }
}
\cs_new:Npn \__krishna_csv_load:nn #1 #2 %#2: Dateiname, #1: Namespace
{
	\ior_new:c {l__krishna_ior_#1_tmpa} % reserves name of IO-Stream
	\ior_open:cn {l__krishna_ior_#1_tmpa} {#2} % load the File into the IO-Stream
	\seq_new:c {l__krishna_ #1 _seq } % l__krishna_ #1_seq, das was vorher als frei registriert wurde
	\prop_new:c {l__krishna_ #1 _prop} % new
	\seq_new:c {l__krishna_  #1 _seq_unit}
	%\prop_new:c {l__krishna_ #1 _prop_unit}
	%\exp_args:Nc \__krishna_csv_read:N { #1 _seq } % fully expanded
	%\ior_str_map_inline:cn {l__krishna_ior_#1_tmpa}  { \seq_put_right:cn {l__krishna_ #1_seq} {##1} }  % new
	\ior_map_inline:cn {l__krishna_ior_#1_tmpa}  { \seq_put_right:cn {l__krishna_ #1_seq} {##1} }  % new

	% TODO Cant currently access Description or name
	% TODO Change the delimiter to \t here, how?
	\seq_map_inline:cn {l__krishna_ #1 _seq}  { \prop_put:cxx {l__krishna_ #1_prop} {\clist_item:nn {##1}{1}} {\clist_item:nn{##1}{2}} }  % new
	\seq_map_inline:cn {l__krishna_ #1 _seq}  { \prop_put:cxx {l__krishna_ #1_prop_unit} {\clist_item:nn {##1}{1}} {\clist_item:nn{##1}{3}} }


	\ior_close:c {l__krishna_ior_#1_tmpa}
}
\cs_new:Npn \__krishna_csv_read:N #1
{
	%\ior_str_map_inline:Nn \l__krishna_ior_tmpa  { \seq_put_right:Nn #1 {##1} }
}
\cs_new:Npn \krishna_csv_get:nnn #1 #2 #3 % not used
{
	\exp_args:Nf
	\clist_item:nn % comma list WIE AUF \t UMSTELLEN?????
	{ \seq_item:cn { l__krishna_ #1 _seq } {#2} }
	{#3}
}
\msg_new:nnn { krishna / csv } { name-used }
{ The~CSV~name~`#1'~is~already~taken }

\newcommand{\mvar}[2] % #1=namespace, #2=variable
{
	\prop_item:cn {l__krishna_ #1 _prop} {#2}
}
\newcommand{\mvarunit}[2]
{
	\prop_item:cn {l__krishna_ #1 _prop_unit} {#2}
}
% Logic function: if value is 1 do true part, otherwise true
\newcommand{\mvaristrue}[4]  %#1=namespace, #2=variablename, #3=text if true, #4=text if false
{
	\int_set:Nn \l_tmpa_int {\int_eval:n {\prop_item:cn {l__krishna_ #1_prop}{#2}}}
	\int_compare:nNnTF {\l_tmpa_int} = {1} {#3} {#4}
}
\newcommand{\mvarsi}[2] %[3]
{
	%\mvar{#1}{#2}\,\si{#3}
	%\SI{\mvar{#1}{#2}}{#3}
	\SI{\mvar{#1}{#2}}{\mvarunit{#1}{#2}}
}
\ExplSyntaxOff


%\loadvariables{t}{test.txt}
%\begin{document}
%	\tb
%\end{document}
%
%	gives pdf with 6.453


\usepackage{expl3}
\usepackage{xparse}
\usepackage{forloop}
\usepackage{siunitx}


\ExplSyntaxOn
% because we want different ior variables for each instance of loadvariables{...} and ior_str_map_inline doesnt natively accept "variable" names 
\cs_generate_variant:Nn \ior_str_map_inline:Nn { c } 
\cs_generate_variant:Nn \ior_map_inline:Nn { c }
\cs_generate_variant:Nn \tl_if_eq:nnTF { V }

\newcommand{\loadvariables}[2]{
	% this is a dummy command to find the reference to transfer files during precompilation. The real work is being done by \_loadvariables below
}

\newcommand{\backgroundloadvariables}[2]{
	\krishna_csv_load:nn {#1} {#2} % 1: Namespace, 2: Filename/Path
}
% Copied from https://tex.stackexchange.com/questions/474397/populate-information-from-a-csv-file-into-a-latex-document-specifically-into-th/474404#474404
% I dont know enough latex3 to do this myself or do it cleaner :)
%\ior_new:N \l__krishna_ior_tmpa
\cs_new:Npn \krishna_csv_load:nn #1 #2 %#2: Dateiname, #1: Name der CSV intern (Namespace)
{
	\cs_if_free:cTF { l__krishna_ #1 _seq }
	% guckt nach, ob \l__krishna_ #1_seq  noch nicht existiert
	%{ \__krishna_csv_load:nn { l__krishna_ #1 } {#2} } % springt immer hier hin, da immer war
	{ \__krishna_csv_load:nn {#1} {#2} } % springt immer hier hin, da immer war
	{ \msg_error:nnn { krishna / csv } { name-used } {#1} }
}
\cs_new:Npn \__krishna_csv_load:nn #1 #2 %#2: Dateiname, #1: Namespace
{
	\ior_new:c {l__krishna_ior_#1_tmpa} % reserves name of IO-Stream
	\ior_open:cn {l__krishna_ior_#1_tmpa} {#2} % load the File into the IO-Stream
	\seq_new:c {l__krishna_ #1 _seq } % l__krishna_ #1_seq, das was vorher als frei registriert wurde
	\prop_new:c {l__krishna_ #1 _prop} % new
	\seq_new:c {l__krishna_  #1 _seq_unit}
	%\prop_new:c {l__krishna_ #1 _prop_unit}
	%\exp_args:Nc \__krishna_csv_read:N { #1 _seq } % fully expanded
	%\ior_str_map_inline:cn {l__krishna_ior_#1_tmpa}  { \seq_put_right:cn {l__krishna_ #1_seq} {##1} }  % new
	\ior_map_inline:cn {l__krishna_ior_#1_tmpa}  { \seq_put_right:cn {l__krishna_ #1_seq} {##1} }  % new

	% TODO Cant currently access Description or name
	% TODO Change the delimiter to \t here, how?
	\seq_map_inline:cn {l__krishna_ #1 _seq}  { \prop_put:cxx {l__krishna_ #1_prop} {\clist_item:nn {##1}{1}} {\clist_item:nn{##1}{2}} }  % new
	\seq_map_inline:cn {l__krishna_ #1 _seq}  { \prop_put:cxx {l__krishna_ #1_prop_unit} {\clist_item:nn {##1}{1}} {\clist_item:nn{##1}{3}} }


	\ior_close:c {l__krishna_ior_#1_tmpa}
}
\cs_new:Npn \__krishna_csv_read:N #1
{
	%\ior_str_map_inline:Nn \l__krishna_ior_tmpa  { \seq_put_right:Nn #1 {##1} }
}
\cs_new:Npn \krishna_csv_get:nnn #1 #2 #3 % not used
{
	\exp_args:Nf
	\clist_item:nn % comma list WIE AUF \t UMSTELLEN?????
	{ \seq_item:cn { l__krishna_ #1 _seq } {#2} }
	{#3}
}
\msg_new:nnn { krishna / csv } { name-used }
{ The~CSV~name~`#1'~is~already~taken }

\newcommand{\mvar}[2] % #1=namespace, #2=variable
{
	\prop_item:cn {l__krishna_ #1 _prop} {#2}
}
\newcommand{\mvarunit}[2]
{
	\prop_item:cn {l__krishna_ #1 _prop_unit} {#2}
}
% Logic function: if value is 1 do true part, otherwise true
\newcommand{\mvaristrue}[4]  %#1=namespace, #2=variablename, #3=text if true, #4=text if false
{
	\int_set:Nn \l_tmpa_int {\int_eval:n {\prop_item:cn {l__krishna_ #1_prop}{#2}}}
	\int_compare:nNnTF {\l_tmpa_int} = {1} {#3} {#4}
}
\newcommand{\mvarsi}[2] %[3]
{
	%\mvar{#1}{#2}\,\si{#3}
	%\SI{\mvar{#1}{#2}}{#3}
	\SI{\mvar{#1}{#2}}{\mvarunit{#1}{#2}}
}
\ExplSyntaxOff



\backgroundloadvariables{m}{./../matlab/test.txt}
\backgroundloadvariables{d}{./test.txt}
\backgroundloadvariables{t}{./unterordner/test6.txt}


% example of cross reference: takes the text.txt from the matlab folder
\loadvariables{m}{../matlab/test.txt} 
\begin{document}
	%\tb
	This is an example for the usage of the mvar system in \LaTeX.
	
	You can import transfer files inside or outside of the \verb|\begin{document}| using \verb|\loadvariables{d}{test.txt}| where d is the namespace given to the loaded variables from the transfer file test.txt. In the source for this pdf  \verb|example.tex| I have done so below:
	\loadvariables{d}{test.txt}
	
	The transferfile \verb|test.txt| contains a variable $tt$ with the value $\mvar{d}{tt}$\footnote{If you want to have it in the correct notation you can encase it in the num command to show $\num{\mvar{d}{tt}}$ } which can also be displayed with its (SI-) unit to $\mvarsi{d}{tt}$. This is being archieved with \verb|\mvar{d}{tt}| and \verb|\mvarsi{d}{tt}|.
	
	That file also contains a symbolic expression (formula)
	\begin{align}
		\mvar{d}{lab}.
	\end{align}
	
	The following text has been loaded from the transfer file: \mvar{d}{cd}
	
	This following variable ($d_3$ from the file \verb|./unterordner/test6.txt|) is a vector directly exported from Matlab:
	$$d_3 = \mvar{t}{d_3}$$
	Its transfer file has been loaded as the namespace $t$ in another file in a subdirectory of this \verb|example.tex| after its usage. This is possible due to the precompilation, which allows for free restructuring of your document without worrying about breaking references.
	\loadvariables{t}{test6.txt}
	
	If you want to use logical building blocks for automated documents, you can use logical variables which uses 1 for True and 0 for False. This can be used with \verb|\mvaristrue{[namespace]}{[var name]}{[text if True]}{[text if False]}|. Example:
	\mvaristrue{t}{logic}{
		This text is printed if the variable $logic$ in the namespace $t$ is 1 (True)
		}{
		This text is printed if the variable $logic$ in the namespace $t$ is 0 (False)
		}
		
	The following list of abbreviations has been automatically generated during precompilation:
	\begin{longtable}{p{.1\textwidth} p{.1\textwidth} p{.1\textwidth} p{.70\textwidth}}
Bezeichner&	Wert&	Einheit&	Beschreibung\\\midrule
a&	1&	-&	-\\\midrule
b&	2.5675&	-&	-\\\midrule
b&	2.568&	-&	-\\\midrule
c&	45&	-&	-\\\midrule
c&	45&	-&	-\\\midrule
cd&	abcgefsdvgfaiijvnbarfigjbn&	-&	-\\\midrule
chr&	\frac{\varphi }{x}+x^2&	-&	-\\\midrule
d_!3&	1234&	-&	-\\\midrule
d_3&	456&	-&	-\\\midrule
d_3&	  \begin{pmatrix}   13.000 \cr   15.000 \cr   149.000 \cr   132.000  \end{pmatrix}&	-&	-\\\midrule
lab&	\frac{1}{x}&	-&	-\\\midrule
t&	0&	-&	-\\\midrule
tt&	1e-05&	-&	-\\\midrule
\end{longtable}
	It only includes numerical variables or abbreviations that don't have a value (the default value -).

\end{document}