% Latex implementation of the variable transfer
% Usage:
% \loadvariables{<namespace>}{<file path>}
% 	<file path> can be (tested on linux):
%		../folder/filename.txt if in another folder (changes directory once and then goes to the folder "folder"
%		filename.txt if in the same folder
% 	<namespace> gives the prefix of all variables from this file
%
%\<namespace><var name>
%	gives the value of the variable with <var name> from the file

%example:
%	given textfile named test.txt in same folder as this file and the main.tex:
% a,45
% b,6.453
%
%	in main.tex:
%\documentclass[11pt,a4paper,sans]{article}
%% Latex implementation of the variable transfer
% Usage:
% \loadvariables{<namespace>}{<file path>}
% 	<file path> can be (tested on linux):
%		../folder/filename.txt if in another folder (changes directory once and then goes to the folder "folder"
%		filename.txt if in the same folder
% 	<namespace> gives the prefix of all variables from this file
%
%\<namespace><var name>
%	gives the value of the variable with <var name> from the file

%example:
%	given textfile named test.txt in same folder as this file and the main.tex:
% a,45
% b,6.453
%
%	in main.tex:
%\documentclass[11pt,a4paper,sans]{article}
%% Latex implementation of the variable transfer
% Usage:
% \loadvariables{<namespace>}{<file path>}
% 	<file path> can be (tested on linux):
%		../folder/filename.txt if in another folder (changes directory once and then goes to the folder "folder"
%		filename.txt if in the same folder
% 	<namespace> gives the prefix of all variables from this file
%
%\<namespace><var name>
%	gives the value of the variable with <var name> from the file

%example:
%	given textfile named test.txt in same folder as this file and the main.tex:
% a,45
% b,6.453
%
%	in main.tex:
%\documentclass[11pt,a4paper,sans]{article}
%% Latex implementation of the variable transfer
% Usage:
% \loadvariables{<namespace>}{<file path>}
% 	<file path> can be (tested on linux):
%		../folder/filename.txt if in another folder (changes directory once and then goes to the folder "folder"
%		filename.txt if in the same folder
% 	<namespace> gives the prefix of all variables from this file
%
%\<namespace><var name>
%	gives the value of the variable with <var name> from the file

%example:
%	given textfile named test.txt in same folder as this file and the main.tex:
% a,45
% b,6.453
%
%	in main.tex:
%\documentclass[11pt,a4paper,sans]{article}
%\include{MLtransfer.tex}
%\loadvariables{t}{test.txt}
%\begin{document}
%	\tb
%\end{document}
%
%	gives pdf with 6.453


\usepackage{expl3}
\usepackage{xparse}
\usepackage{forloop}
\usepackage{siunitx}


\ExplSyntaxOn
% because we want different ior variables for each instance of loadvariables{...} and ior_str_map_inline doesnt natively accept "variable" names 
\cs_generate_variant:Nn \ior_str_map_inline:Nn { c } 
\cs_generate_variant:Nn \ior_map_inline:Nn { c }
\cs_generate_variant:Nn \tl_if_eq:nnTF { V }

\newcommand{\loadvariables}[2]{
	% this is a dummy command to find the reference to transfer files during precompilation. The real work is being done by \_loadvariables below
}

\newcommand{\backgroundloadvariables}[2]{
	\krishna_csv_load:nn {#1} {#2} % 1: Namespace, 2: Filename/Path
}
% Copied from https://tex.stackexchange.com/questions/474397/populate-information-from-a-csv-file-into-a-latex-document-specifically-into-th/474404#474404
% I dont know enough latex3 to do this myself or do it cleaner :)
%\ior_new:N \l__krishna_ior_tmpa
\cs_new:Npn \krishna_csv_load:nn #1 #2 %#2: Dateiname, #1: Name der CSV intern (Namespace)
{
	\cs_if_free:cTF { l__krishna_ #1 _seq }
	% guckt nach, ob \l__krishna_ #1_seq  noch nicht existiert
	%{ \__krishna_csv_load:nn { l__krishna_ #1 } {#2} } % springt immer hier hin, da immer war
	{ \__krishna_csv_load:nn {#1} {#2} } % springt immer hier hin, da immer war
	{ \msg_error:nnn { krishna / csv } { name-used } {#1} }
}
\cs_new:Npn \__krishna_csv_load:nn #1 #2 %#2: Dateiname, #1: Namespace
{
	\ior_new:c {l__krishna_ior_#1_tmpa} % reserves name of IO-Stream
	\ior_open:cn {l__krishna_ior_#1_tmpa} {#2} % load the File into the IO-Stream
	\seq_new:c {l__krishna_ #1 _seq } % l__krishna_ #1_seq, das was vorher als frei registriert wurde
	\prop_new:c {l__krishna_ #1 _prop} % new
	\seq_new:c {l__krishna_  #1 _seq_unit}
	%\prop_new:c {l__krishna_ #1 _prop_unit}
	%\exp_args:Nc \__krishna_csv_read:N { #1 _seq } % fully expanded
	%\ior_str_map_inline:cn {l__krishna_ior_#1_tmpa}  { \seq_put_right:cn {l__krishna_ #1_seq} {##1} }  % new
	\ior_map_inline:cn {l__krishna_ior_#1_tmpa}  { \seq_put_right:cn {l__krishna_ #1_seq} {##1} }  % new

	% TODO Cant currently access Description or name
	% TODO Change the delimiter to \t here, how?
	\seq_map_inline:cn {l__krishna_ #1 _seq}  { \prop_put:cxx {l__krishna_ #1_prop} {\clist_item:nn {##1}{1}} {\clist_item:nn{##1}{2}} }  % new
	\seq_map_inline:cn {l__krishna_ #1 _seq}  { \prop_put:cxx {l__krishna_ #1_prop_unit} {\clist_item:nn {##1}{1}} {\clist_item:nn{##1}{3}} }


	\ior_close:c {l__krishna_ior_#1_tmpa}
}
\cs_new:Npn \__krishna_csv_read:N #1
{
	%\ior_str_map_inline:Nn \l__krishna_ior_tmpa  { \seq_put_right:Nn #1 {##1} }
}
\cs_new:Npn \krishna_csv_get:nnn #1 #2 #3 % not used
{
	\exp_args:Nf
	\clist_item:nn % comma list WIE AUF \t UMSTELLEN?????
	{ \seq_item:cn { l__krishna_ #1 _seq } {#2} }
	{#3}
}
\msg_new:nnn { krishna / csv } { name-used }
{ The~CSV~name~`#1'~is~already~taken }

\newcommand{\mvar}[2] % #1=namespace, #2=variable
{
	\prop_item:cn {l__krishna_ #1 _prop} {#2}
}
\newcommand{\mvarunit}[2]
{
	\prop_item:cn {l__krishna_ #1 _prop_unit} {#2}
}
% Logic function: if value is 1 do true part, otherwise true
\newcommand{\mvaristrue}[4]  %#1=namespace, #2=variablename, #3=text if true, #4=text if false
{
	\int_set:Nn \l_tmpa_int {\int_eval:n {\prop_item:cn {l__krishna_ #1_prop}{#2}}}
	\int_compare:nNnTF {\l_tmpa_int} = {1} {#3} {#4}
}
\newcommand{\mvarsi}[2] %[3]
{
	%\mvar{#1}{#2}\,\si{#3}
	%\SI{\mvar{#1}{#2}}{#3}
	\SI{\mvar{#1}{#2}}{\mvarunit{#1}{#2}}
}
\ExplSyntaxOff


%\loadvariables{t}{test.txt}
%\begin{document}
%	\tb
%\end{document}
%
%	gives pdf with 6.453


\usepackage{expl3}
\usepackage{xparse}
\usepackage{forloop}
\usepackage{siunitx}


\ExplSyntaxOn
% because we want different ior variables for each instance of loadvariables{...} and ior_str_map_inline doesnt natively accept "variable" names 
\cs_generate_variant:Nn \ior_str_map_inline:Nn { c } 
\cs_generate_variant:Nn \ior_map_inline:Nn { c }
\cs_generate_variant:Nn \tl_if_eq:nnTF { V }

\newcommand{\loadvariables}[2]{
	% this is a dummy command to find the reference to transfer files during precompilation. The real work is being done by \_loadvariables below
}

\newcommand{\backgroundloadvariables}[2]{
	\krishna_csv_load:nn {#1} {#2} % 1: Namespace, 2: Filename/Path
}
% Copied from https://tex.stackexchange.com/questions/474397/populate-information-from-a-csv-file-into-a-latex-document-specifically-into-th/474404#474404
% I dont know enough latex3 to do this myself or do it cleaner :)
%\ior_new:N \l__krishna_ior_tmpa
\cs_new:Npn \krishna_csv_load:nn #1 #2 %#2: Dateiname, #1: Name der CSV intern (Namespace)
{
	\cs_if_free:cTF { l__krishna_ #1 _seq }
	% guckt nach, ob \l__krishna_ #1_seq  noch nicht existiert
	%{ \__krishna_csv_load:nn { l__krishna_ #1 } {#2} } % springt immer hier hin, da immer war
	{ \__krishna_csv_load:nn {#1} {#2} } % springt immer hier hin, da immer war
	{ \msg_error:nnn { krishna / csv } { name-used } {#1} }
}
\cs_new:Npn \__krishna_csv_load:nn #1 #2 %#2: Dateiname, #1: Namespace
{
	\ior_new:c {l__krishna_ior_#1_tmpa} % reserves name of IO-Stream
	\ior_open:cn {l__krishna_ior_#1_tmpa} {#2} % load the File into the IO-Stream
	\seq_new:c {l__krishna_ #1 _seq } % l__krishna_ #1_seq, das was vorher als frei registriert wurde
	\prop_new:c {l__krishna_ #1 _prop} % new
	\seq_new:c {l__krishna_  #1 _seq_unit}
	%\prop_new:c {l__krishna_ #1 _prop_unit}
	%\exp_args:Nc \__krishna_csv_read:N { #1 _seq } % fully expanded
	%\ior_str_map_inline:cn {l__krishna_ior_#1_tmpa}  { \seq_put_right:cn {l__krishna_ #1_seq} {##1} }  % new
	\ior_map_inline:cn {l__krishna_ior_#1_tmpa}  { \seq_put_right:cn {l__krishna_ #1_seq} {##1} }  % new

	% TODO Cant currently access Description or name
	% TODO Change the delimiter to \t here, how?
	\seq_map_inline:cn {l__krishna_ #1 _seq}  { \prop_put:cxx {l__krishna_ #1_prop} {\clist_item:nn {##1}{1}} {\clist_item:nn{##1}{2}} }  % new
	\seq_map_inline:cn {l__krishna_ #1 _seq}  { \prop_put:cxx {l__krishna_ #1_prop_unit} {\clist_item:nn {##1}{1}} {\clist_item:nn{##1}{3}} }


	\ior_close:c {l__krishna_ior_#1_tmpa}
}
\cs_new:Npn \__krishna_csv_read:N #1
{
	%\ior_str_map_inline:Nn \l__krishna_ior_tmpa  { \seq_put_right:Nn #1 {##1} }
}
\cs_new:Npn \krishna_csv_get:nnn #1 #2 #3 % not used
{
	\exp_args:Nf
	\clist_item:nn % comma list WIE AUF \t UMSTELLEN?????
	{ \seq_item:cn { l__krishna_ #1 _seq } {#2} }
	{#3}
}
\msg_new:nnn { krishna / csv } { name-used }
{ The~CSV~name~`#1'~is~already~taken }

\newcommand{\mvar}[2] % #1=namespace, #2=variable
{
	\prop_item:cn {l__krishna_ #1 _prop} {#2}
}
\newcommand{\mvarunit}[2]
{
	\prop_item:cn {l__krishna_ #1 _prop_unit} {#2}
}
% Logic function: if value is 1 do true part, otherwise true
\newcommand{\mvaristrue}[4]  %#1=namespace, #2=variablename, #3=text if true, #4=text if false
{
	\int_set:Nn \l_tmpa_int {\int_eval:n {\prop_item:cn {l__krishna_ #1_prop}{#2}}}
	\int_compare:nNnTF {\l_tmpa_int} = {1} {#3} {#4}
}
\newcommand{\mvarsi}[2] %[3]
{
	%\mvar{#1}{#2}\,\si{#3}
	%\SI{\mvar{#1}{#2}}{#3}
	\SI{\mvar{#1}{#2}}{\mvarunit{#1}{#2}}
}
\ExplSyntaxOff


%\loadvariables{t}{test.txt}
%\begin{document}
%	\tb
%\end{document}
%
%	gives pdf with 6.453


\usepackage{expl3}
\usepackage{xparse}
\usepackage{forloop}
\usepackage{siunitx}


\ExplSyntaxOn
% because we want different ior variables for each instance of loadvariables{...} and ior_str_map_inline doesnt natively accept "variable" names 
\cs_generate_variant:Nn \ior_str_map_inline:Nn { c } 
\cs_generate_variant:Nn \ior_map_inline:Nn { c }
\cs_generate_variant:Nn \tl_if_eq:nnTF { V }

\newcommand{\loadvariables}[2]{
	% this is a dummy command to find the reference to transfer files during precompilation. The real work is being done by \_loadvariables below
}

\newcommand{\backgroundloadvariables}[2]{
	\krishna_csv_load:nn {#1} {#2} % 1: Namespace, 2: Filename/Path
}
% Copied from https://tex.stackexchange.com/questions/474397/populate-information-from-a-csv-file-into-a-latex-document-specifically-into-th/474404#474404
% I dont know enough latex3 to do this myself or do it cleaner :)
%\ior_new:N \l__krishna_ior_tmpa
\cs_new:Npn \krishna_csv_load:nn #1 #2 %#2: Dateiname, #1: Name der CSV intern (Namespace)
{
	\cs_if_free:cTF { l__krishna_ #1 _seq }
	% guckt nach, ob \l__krishna_ #1_seq  noch nicht existiert
	%{ \__krishna_csv_load:nn { l__krishna_ #1 } {#2} } % springt immer hier hin, da immer war
	{ \__krishna_csv_load:nn {#1} {#2} } % springt immer hier hin, da immer war
	{ \msg_error:nnn { krishna / csv } { name-used } {#1} }
}
\cs_new:Npn \__krishna_csv_load:nn #1 #2 %#2: Dateiname, #1: Namespace
{
	\ior_new:c {l__krishna_ior_#1_tmpa} % reserves name of IO-Stream
	\ior_open:cn {l__krishna_ior_#1_tmpa} {#2} % load the File into the IO-Stream
	\seq_new:c {l__krishna_ #1 _seq } % l__krishna_ #1_seq, das was vorher als frei registriert wurde
	\prop_new:c {l__krishna_ #1 _prop} % new
	\seq_new:c {l__krishna_  #1 _seq_unit}
	%\prop_new:c {l__krishna_ #1 _prop_unit}
	%\exp_args:Nc \__krishna_csv_read:N { #1 _seq } % fully expanded
	%\ior_str_map_inline:cn {l__krishna_ior_#1_tmpa}  { \seq_put_right:cn {l__krishna_ #1_seq} {##1} }  % new
	\ior_map_inline:cn {l__krishna_ior_#1_tmpa}  { \seq_put_right:cn {l__krishna_ #1_seq} {##1} }  % new

	% TODO Cant currently access Description or name
	% TODO Change the delimiter to \t here, how?
	\seq_map_inline:cn {l__krishna_ #1 _seq}  { \prop_put:cxx {l__krishna_ #1_prop} {\clist_item:nn {##1}{1}} {\clist_item:nn{##1}{2}} }  % new
	\seq_map_inline:cn {l__krishna_ #1 _seq}  { \prop_put:cxx {l__krishna_ #1_prop_unit} {\clist_item:nn {##1}{1}} {\clist_item:nn{##1}{3}} }


	\ior_close:c {l__krishna_ior_#1_tmpa}
}
\cs_new:Npn \__krishna_csv_read:N #1
{
	%\ior_str_map_inline:Nn \l__krishna_ior_tmpa  { \seq_put_right:Nn #1 {##1} }
}
\cs_new:Npn \krishna_csv_get:nnn #1 #2 #3 % not used
{
	\exp_args:Nf
	\clist_item:nn % comma list WIE AUF \t UMSTELLEN?????
	{ \seq_item:cn { l__krishna_ #1 _seq } {#2} }
	{#3}
}
\msg_new:nnn { krishna / csv } { name-used }
{ The~CSV~name~`#1'~is~already~taken }

\newcommand{\mvar}[2] % #1=namespace, #2=variable
{
	\prop_item:cn {l__krishna_ #1 _prop} {#2}
}
\newcommand{\mvarunit}[2]
{
	\prop_item:cn {l__krishna_ #1 _prop_unit} {#2}
}
% Logic function: if value is 1 do true part, otherwise true
\newcommand{\mvaristrue}[4]  %#1=namespace, #2=variablename, #3=text if true, #4=text if false
{
	\int_set:Nn \l_tmpa_int {\int_eval:n {\prop_item:cn {l__krishna_ #1_prop}{#2}}}
	\int_compare:nNnTF {\l_tmpa_int} = {1} {#3} {#4}
}
\newcommand{\mvarsi}[2] %[3]
{
	%\mvar{#1}{#2}\,\si{#3}
	%\SI{\mvar{#1}{#2}}{#3}
	\SI{\mvar{#1}{#2}}{\mvarunit{#1}{#2}}
}
\ExplSyntaxOff


%\loadvariables{t}{test.txt}
%\begin{document}
%	\tb
%\end{document}
%
%	gives pdf with 6.453


\usepackage{expl3}
\usepackage{xparse}
\usepackage{forloop}
\usepackage{siunitx}


\ExplSyntaxOn
% because we want different ior variables for each instance of loadvariables{...} and ior_str_map_inline doesnt natively accept "variable" names 
\cs_generate_variant:Nn \ior_str_map_inline:Nn { c } 
\cs_generate_variant:Nn \ior_map_inline:Nn { c }
\cs_generate_variant:Nn \tl_if_eq:nnTF { V }

\newcommand{\loadvariables}[2]{
	% this is a dummy command to find the reference to transfer files during precompilation. The real work is being done by \_loadvariables below
}

\newcommand{\backgroundloadvariables}[2]{
	\krishna_csv_load:nn {#1} {#2} % 1: Namespace, 2: Filename/Path
}
% Copied from https://tex.stackexchange.com/questions/474397/populate-information-from-a-csv-file-into-a-latex-document-specifically-into-th/474404#474404
% I dont know enough latex3 to do this myself or do it cleaner :)
%\ior_new:N \l__krishna_ior_tmpa
\cs_new:Npn \krishna_csv_load:nn #1 #2 %#2: Dateiname, #1: Name der CSV intern (Namespace)
{
	\cs_if_free:cTF { l__krishna_ #1 _seq }
	% guckt nach, ob \l__krishna_ #1_seq  noch nicht existiert
	%{ \__krishna_csv_load:nn { l__krishna_ #1 } {#2} } % springt immer hier hin, da immer war
	{ \__krishna_csv_load:nn {#1} {#2} } % springt immer hier hin, da immer war
	{ \msg_error:nnn { krishna / csv } { name-used } {#1} }
}
\cs_new:Npn \__krishna_csv_load:nn #1 #2 %#2: Dateiname, #1: Namespace
{
	\ior_new:c {l__krishna_ior_#1_tmpa} % reserves name of IO-Stream
	\ior_open:cn {l__krishna_ior_#1_tmpa} {#2} % load the File into the IO-Stream
	\seq_new:c {l__krishna_ #1 _seq } % l__krishna_ #1_seq, das was vorher als frei registriert wurde
	\prop_new:c {l__krishna_ #1 _prop} % new
	\seq_new:c {l__krishna_  #1 _seq_unit}
	%\prop_new:c {l__krishna_ #1 _prop_unit}
	%\exp_args:Nc \__krishna_csv_read:N { #1 _seq } % fully expanded
	%\ior_str_map_inline:cn {l__krishna_ior_#1_tmpa}  { \seq_put_right:cn {l__krishna_ #1_seq} {##1} }  % new
	\ior_map_inline:cn {l__krishna_ior_#1_tmpa}  { \seq_put_right:cn {l__krishna_ #1_seq} {##1} }  % new

	% TODO Cant currently access Description or name
	% TODO Change the delimiter to \t here, how?
	\seq_map_inline:cn {l__krishna_ #1 _seq}  { \prop_put:cxx {l__krishna_ #1_prop} {\clist_item:nn {##1}{1}} {\clist_item:nn{##1}{2}} }  % new
	\seq_map_inline:cn {l__krishna_ #1 _seq}  { \prop_put:cxx {l__krishna_ #1_prop_unit} {\clist_item:nn {##1}{1}} {\clist_item:nn{##1}{3}} }


	\ior_close:c {l__krishna_ior_#1_tmpa}
}
\cs_new:Npn \__krishna_csv_read:N #1
{
	%\ior_str_map_inline:Nn \l__krishna_ior_tmpa  { \seq_put_right:Nn #1 {##1} }
}
\cs_new:Npn \krishna_csv_get:nnn #1 #2 #3 % not used
{
	\exp_args:Nf
	\clist_item:nn % comma list WIE AUF \t UMSTELLEN?????
	{ \seq_item:cn { l__krishna_ #1 _seq } {#2} }
	{#3}
}
\msg_new:nnn { krishna / csv } { name-used }
{ The~CSV~name~`#1'~is~already~taken }

\newcommand{\mvar}[2] % #1=namespace, #2=variable
{
	\prop_item:cn {l__krishna_ #1 _prop} {#2}
}
\newcommand{\mvarunit}[2]
{
	\prop_item:cn {l__krishna_ #1 _prop_unit} {#2}
}
% Logic function: if value is 1 do true part, otherwise true
\newcommand{\mvaristrue}[4]  %#1=namespace, #2=variablename, #3=text if true, #4=text if false
{
	\int_set:Nn \l_tmpa_int {\int_eval:n {\prop_item:cn {l__krishna_ #1_prop}{#2}}}
	\int_compare:nNnTF {\l_tmpa_int} = {1} {#3} {#4}
}
\newcommand{\mvarsi}[2] %[3]
{
	%\mvar{#1}{#2}\,\si{#3}
	%\SI{\mvar{#1}{#2}}{#3}
	\SI{\mvar{#1}{#2}}{\mvarunit{#1}{#2}}
}
\ExplSyntaxOff

